\chapter{Conclusioni e sviluppi futuri}
In questo capitolo finale tratteremo i possibili sviluppi futuri che possono essere apportati al sistema realizzato, al fine di migliorarne le prestazioni sia in termini di risultati, sia in termini di efficienza del codice.

\section{Conclusioni}
Prima di procedere illustrando i possibili sviluppi della rete è bene ricapitolare brevemente le conclusioni dedotte dai risultati commentati nei capitoli precedenti.
\\Come prima cosa, la più importante, possiamo dire di aver ottenuto dei risultati pienamente soddisfacenti. Lo scopo del sistema, infatti, era quello di migliorare la CycleGan proposta dal Paper \cite{Zhu_2017_ICCV} grazie al trasferimento dell'attenzione durante l'addestramento. Scopo che, come ampiamente trattato nel capitolo precedente, è stato portato a termine. Abbiamo visto, infatti, come i risultati della rete che sfrutta GradCam sull'ultimo blocco residuo siano migliori rispetto alla rete \emph{standard}. Abbiamo ottenuto dei risultati migliori applicando GradCam su entrambi i dataset utilizzati. 